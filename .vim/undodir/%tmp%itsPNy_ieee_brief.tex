Vim�UnDo�k������erâ��:QXT�F�ެXQ�4�R]Rest of the paper is organized as follows. Section \ref{sec:background} covers the background of DNNs and stuck-at faults in emerging NVMs. Related works are presented in Section \ref{sec:related}. Section \ref{sec:previous-proposed-techniques} introduce the Intra-block Address Remapping and Weight Inversion techniques. The proposed Critical-Aware Weight Rotation technique is introduced in Section \ref{sec:proposed-technique}. Finally, we evaluates the effectiveness of the proposed methods and existing techniques in Section \ref{sec:evaluation} and Section \ref{sec:conclusion} concludes the paper. `�bM7t_�d�����bM7V�ceLArtificial neural network (ANNs), discovered back in the 1940s, are the computer algorithms inspired by the biological brains of animal. A layer of an ANN often consists of multiple nodes (i.e., neurons) connected to next layer by multiple connections (i.e., synapses/weights). Typical ANNs is made up of an input layer, an output layer and multiple hidden layers in between. A subset of ANN, Deep Neural Network (DNN), is an ANN which has a large amount of hidden layers (hence the name \textit{"Deep"} Neural Network). Over the last decade, DNNs have made a major breakthrough in the field of computer vision and natural language processing, making conventional programming approaches obsolete. However, despite the great performance, DNNs requires immense amount of hardware to store their weights and biases. This makes it difficult for deploying large-scale DNN on resource-constrained hardware like mobile devices. To tackle this problem of DNNs, several techniques have been proposed to reduce the network size making DNNs easier to deploy \cite{han2015deep}, \cite{krishnamoorthi2018quantizing}. However, the cost of using these techniques is that of accuracy loss of DNNs. Depending on applications, this accuracy loss may or may not be acceptable. Along with such approaches, many studies have been investigated on emerging non-volatile memory (NVMs) technology to provide high-bandwidth, scalable and non-volatile platform for deploying DNNs. With such advantages over traditional charge-based memory, emerging NVMs is sought to be an ideal candidate for employing efficient and high-performance DNNs.5�_�`�����bM7s�_aR]Rest of the paper is organized as follows. Section \ref{sec:background} covers the background of DNNs and stuck-at faults in emerging NVMs. Related works are presented in Section \ref{sec:related}. Section \ref{sec:previous-proposed-techniques} introduce the Intra-block Address Remapping and Weight Inversion techniques. The proposed Critical-Aware Weight Rotation technique is introduced in Section \ref{sec:proposed-technique}. Finally, we evaluates the effectiveness of the proposed methods and existing techniques in Section \ref{sec:evaluation} and Section \ref{sec:conclusion} concludes the paper. 5��